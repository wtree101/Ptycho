\documentclass{article}
\usepackage[utf8]{inputenc}
%\usepackage{biblatex}
\usepackage{amssymb}
\usepackage{amsmath}
\usepackage{amsthm}
\usepackage{graphicx}
\usepackage{float}
\usepackage{url}
\usepackage{framed}
\usepackage{booktabs}
\usepackage{enumitem}
\usepackage{extarrows}
\usepackage{subcaption}
\usepackage{epstopdf}
\usepackage{hyperref}
%\usepackage{algorithm}
%\usepackage{algorithmic}
\usepackage[ruled,linesnumbered]{algorithm2e}
\numberwithin{equation}{section}
%\usepackage{BOONDOX-calo}
\title{Research Proposal: Partially coherent ptychography}
%\author{huibinchang }
\date{July 2021}

\begin{document}

\maketitle
\tableofcontents
\section{Introduction}
%Please describe your proposed research topic (up to 500 characters including spaces).
%The description should include the general field of the research and the specific research question(s).

Ptychography is a popular imaging technique in scientific fields as diverse as condensed matter physics, cell biology, materials science and electronics, among others. In a coherent ptychography experiment, a localized coherent X-ray probe (or illumination) scans through an specimen, while the detector collects a sequence of phaseless intensities in the far field. The goal is to obtain a high resolution reconstruction of the specimen from the sequence of intensity measurements. 

Coherent ptychographic imaging experiments often rely on apertures to define a coherent illumination. Research institutions around the world are investing considerable resources to produce brighter x-ray sources in order to overcome this limitation. Meanwhile, most of the x-ray photons generated are currently discarded by secondary apertures. Even when there is enough coherent flux, the stability required during an exposure is often another limiting factor. In a word, coherent light sources need strict experiment conditions and could cause waste. Both flux and stability limitations can be reduced using partial coherence analysis. 


%Ptychography is a popular imaging technique in scientific fields like condensed matter physics, cell biology, materials science, and electronics, etc. Its goal is to obtain a high-resolution reconstruction of the specimen from the sequence of intensity measurements. Traditional Ptychographic imaging experiments often rely on apertures to define a coherent illumination, which is strict and could cause waste. Both flux and stability limitations can be reduced using partial coherence analysis. 

\section{Objectives}

Generally speaking, we would like to characterize partially coherent to Mathematical language and design an effective algorithm to solve the problem. To prove the rationale of the model and algorithm, quantitative analysis is required to characterize the approximation error of the model and the convergence speed of the algorithm, under suitable assumptions for the phobe and the vibration kernel for a partially coherent effect.





\section{Methods}
\begin{enumerate}
\item Model

Models would be borrowed from physics literature and transformed into mathematical language. Because various models are used to characterize partially coherent effects in different settings, we would like to build connections between models through applied analysis skills. To obtain a high-resolution reconstruction of the specimen from the sequence of intensity measurements, we need to solve an inverse problem, which would be described as an optimization problem. 

\item Algorithm 

There are plenty of optimization algorithms available, like the Gradient descent method.
Considering the non-convex and low-rank nature of this problem, we would focus on algorithms in these fields \cite{lowrank}\cite{ADMM}, and make innovative adjustments utilizing the structure inside the specific model.

\item Experiment

The algorithm would first be tested on simulation data. Then, we would get data from SLAC National Accelerator Laboratory and test on real-world data.

\item Convergence Analysis

For non-convex optimization, we could follow the general framework in \cite{nonconvex}.



\end{enumerate}

\section{Background/prior work}

\begin{enumerate}

\item Model 

We would mainly investigate three models. \cite{mix} proposes a general model based on quantum state tomography. The phobe is assumed to be in a mixed state to represent a partially coherent effect. \cite{psf} and vibration model \cite{chang} are two specific ones.

We have shown that \cite{chang} in a special case of \cite{mix}. Though numerical experiment shows that vibration model is approximated low-rank, no theoretical analysis has been conducted, and the suitable number of states remains empirical. Besides, the decomposed modes are amazingly beautiful, some of which are similar to derivations of the main mode. We used Functional expansion skills like Taylor expansion to expand the phobe under some smooth conditions and got primary explanations.  

\item Problem solving

 ADMM algorithm has been used to solve coherent Ptychography problem with convergence analysis\cite{admm}. In a partially coherent problem, an intuitive AP(alternative projection) algorithm was commonly used. We firstly extended the ADMM algorithm to mixed states, and then tried adjustments to supplement the searching process, like adding orthogonal constraints.
 
\item Experiment

 We used a general model and extended ADMM algorithm, conducting Experiments similar to \cite{chang}. Our methods could overcome larger partially coherent effects, and experiment results show greater speed over AP.

\section{intellectual vision and aspirations}
Ptychography is a field closely integrated with physics. I have read a lot of physics-related literature in my research. From the beginning, I felt a "language barrier" and was difficult to understand, then I gradually realized the simplicity and beauty. Applied mathematics often requires interdisciplinary research to make good and meaningful work. This research is a good attempt and start.

Ptychography is significant in physics and mathematics theory, which can be regarded as a measurement method. If the mixed state is regarded as a black box, then the measurement data generated by Ptychography is to detect this black box. To solve the partially coherent Ptychography problem, physically speaking, we want to characterize a mixed state in quantum physics. Mathematically speaking, we are going to restore a low-rank matrix. Quantum tomography and low-rank matrix restoration are different angles of similar goals. On the one hand, we borrowed the existing mathematical optimization algorithms to solve the problem here. On the other hand, we got inspiration from physics problems and proposed new theories and algorithms for restoring low-rank matrices.

From the application point of view, the characterization of the biased theory can greatly relax the strict requirements on the coherent light sources, and provide new ideas for the development of imaging equipment. It is foreseeable that as the requirements for light sources are relaxed, the cost of light sources (energy consumption, etc.) will be greatly reduced. As the defects in the light source are allowed, the imaging stability is greatly improved, and high-quality images can be obtained under slightly worse conditions. High-precision imaging is indispensable for the exploration of microstructures (physical particles, etc.).

\end{enumerate}

\begin{thebibliography}{99}
\bibitem{chang}{Chang, Huibin, et al. "Partially coherent ptychography by gradient decomposition of the probe." Acta Crystallographica Section A: Foundations and Advances 74.3 (2018): 157-169.}
\bibitem{theory}{Wolf E. New theory of partial coherence in the space–frequency domain. Part I: spectra and cross spectra of steady-state sources[J]. JOSA, 1982, 72(3): 343-351.}
\bibitem{mix}{Thibault P, Menzel A. Reconstructing state mixtures from diffraction measurements[J]. Nature, 2013, 494(7435): 68-71.}
\bibitem{direct}{Multiplexed coded illumination for Fourier Ptychography with an LED array microscope.}
\bibitem{algorithm}{Thibault P, Dierolf M, Bunk O, et al. Probe retrieval in ptychographic coherent diffractive imaging[J]. Ultramicroscopy, 2009, 109(4): 338-343.}
\bibitem{quan}{Introduction to Quantum Mechanics, David J. Griffiths, 12.3}
\bibitem{all}{Fannjiang A, Strohmer T. The numerics of phase retrieval[J]. Acta Numerica, 2020, 29: 125-228.}
\bibitem{admm}{Chang, Huibin, Pablo Enfedaque, and Stefano Marchesini. "Blind ptychographic phase retrieval via convergent alternating direction method of multipliers." SIAM Journal on Imaging Sciences 12.1 (2019): 153-185.}
\bibitem{psf}{Konijnenberg S. An introduction to the theory of ptychographic phase retrieval methods[J]. Advanced Optical Technologies, 2017, 6(6): 423-438.}

\bibitem{nonconvex}{Attouch H, Bolte J, Redont P, et al. Proximal alternating minimization and projection methods for nonconvex problems: An approach based on the Kurdyka-Łojasiewicz inequality[J]. Mathematics of operations research, 2010, 35(2): 438-457.}

\bibitem{lowrank}{Candes E J, Strohmer T, Voroninski V. Phaselift: Exact and stable signal recovery from magnitude measurements via convex programming[J]. Communications on Pure and Applied Mathematics, 2013, 66(8): 1241-1274.}

\bibitem{ADMM}{Boyd S, Parikh N, Chu E. Distributed optimization and statistical learning via the alternating direction method of multipliers[M]. Now Publishers Inc, 2011.}



\end{thebibliography}

I have a strong interest in information science and computational mathematics, especially in digital signal and image processing, and their applications in medicine and brain science. I hope to study for a Ph.D. in this field and make meaningful contributions both in theory and application. I choose this field because it is what I am good at and what I love. I learned to implement data structures and algorithms in C++ from the first grade of junior high school. I participated in the National Youth Computer Competition on behalf of our senior middle school and won the first prize of the National Olympiad in Informatics in Provinces(NOIP). I got a solid computing foundation through six years of training in programming and algorithm. However, I was not satisfied with the engineering-oriented training provided by the Department of Computer Science. I hoped that I could study algorithms and theories more deeply and changed to the Department of Mathematics. Mathematics has indeed given me a lot of surprises. I am obsessed with the abstraction and simplicity of mathematical language.

As for why I am interested in medicine and neroscience, my interest comes from pain. My
family was hit hard by the diseases. I lost my father in junior high school, and my mother-in-law was admitted to the hospital because of schizophrenia in 2020. I think that brain waves, CT, and MRI images, which are often referenced in medical research and diagnosis, are one-dimensional and two-dimensional signals. I strongly hope that I can make a contribution to solving the mechanism of mental illness using the tools from computational mathematics.

At present, I prefer to solve problems from the perspective of theoretical analysis and engage in cutting-edge work. So my first choice is to enter the academia to discover and solve unknown problems. If I go to the industry, I hope to enter some laboratories, at least in highly technical positions such as algorithm engineers (such as Huawei's health monitoring algorithm design for Huawei Bands). My research would be conducive to understanding how to make full use of medical data (physiological electrical signals, medical images), develop new medical devices, and contribute to human health. These cutting-edge technologies can also create huge economic benefits in enterprises.

Apart from academic research, I am curious about the world. I often travel alone or with friends, and have been to Shanghai, Xiamen, Hong Kong, Changsha and other places. I really enjoy making plans by myself and exploring a city. In order to understand the real situation in my country's developing areas, in the summer of 2019, I went to the Buyi ethnic minority settlement in Wangmo County, Guizhou Province for an expedition. We need to set up home visits, communicate and get information from parents and children living in poor there, who we may find funders to help them later on. The mountain road is very rugged and the transportation is inconvenient. The children there are skilled in riding motorcycles to school, and they also drive us to their homes for home visits. Sometimes the schedule is tight and only eat two meals a day, but the children there are habitually having two meals a day. We stayed in their dormitory at night, and when we looked up, we saw moths on the ceiling. We are sad to see that many children hesitate to continue studying because of family difficulties, and we are also gratified to see that many of the country's poverty alleviation policies are indeed beneficial to them. I’m very happy that through seeing with my own eyes, I am closer to the reality of this country.

Besides, I would like to communicate with foreign friends in the Internet. Cambly and Tandem are two apps I like very much. On Cambly, I met foreign teachers from all over the world; on Tandem, I met many foreign friends who were learning Chinese, and we exchanged languages. Whenever I meet my colleagues who major in mathematics and physics, I always feel very kind, and there are always endless topics to talk with them-it fully demonstrates that mathematics is the most common language in the world. For example, a little boy studying artificial intelligence at the University of Edinburgh always asks me some linear algebra questions. A Spanish brother who was studying for a PhD in Physics in Texas, the United States told me that he learned Chinese because he found more and more Chinese materials in the physics literature and believed that Chinese was the future trend. Many points of view are very fresh to me, and I hope to have the opportunity to witness it with my own eyes in the future.

In daily life, I like drawing, reading, and all kinds of sports, like basketball, volleyball, and swimming. I believe that these hobbies are an indispensable component of a complete person, allowing me to love life with a smile even in the most difficult situations. During the COVID, I took classes online at home. My mother-in-law was admitted to the hospital due to schizophrenia and needs family members to take turns to accompany her. I recorded each prescription of the doctor, observed the abnormal reaction after the medication, and communicated with the attending doctor. When things can’t be solved, clinging to this makes people think extreme and pessimistic. On the contrary, when I took up a paintbrush and painted cute little penguins, I drew more than 100 pictures one by one, and I was protected by my own small world. The pain may not be resolved, but the situation will stabilize, at least we can get used to it.

In daily life, I like drawing, reading, and all kinds of sports, like basketball, volleyball, and swimming. In my opinion, these hobbies are an indispensable component of a well-rounded person. First of all, they allow me to love life with a smile even in the most difficult situations. During the COVID, I took classes online at home. My mother-in-law was admitted to the hospital due to schizophrenia and needed family members to take turns to accompany her. I recorded each prescription of the doctor, observed the abnormal reaction after the medication, and communicated with the attending doctor. When things can’t be solved, clinging to this makes people think extreme and pessimistic. On the contrary, when I took up a paintbrush and painted a cute little penguin, I drew almost 100 pictures one by one, and I was protected by my own small world. stand up. The pain may not be resolved, but the situation will stabilize, at least we can get used to it.

And, the precious friendship they brought me. When I first entered the university, due to the school’s policy, our geology major had only freshmen on our campus and no basketball team. But I especially wanted to play basketball, so I co-founded a team with the Ocean Engineering Department and became the first captain. Contacting coaches, organizing training, registering for competitions... everything is a new experience. Our only freshman team has completed many games firmly on the field, often with only 5 players and 1 substitute playing four quarters under the scorching sun. Faced with opponents that often consist of four grades, we have won. Victory is so precious. Later, I handed over the captain to a very powerful forward in the team. This team continues to develop and grow in Zhuhai, and I am proud of them. I especially like ball team sports, because I like to cooperate with the players, and the feeling of winning together, and the friendship of teammates is always pure and vivid.

I am full of curiosity about the world, full of love for my profession, and full of gratitude for life. Therefore, I will never stop exploring, I am not tired of research, and calmly face the difficulties of life. I am proud of myself.


%version2
I have a strong interest in information science and computational mathematics, especially in digital signal and image processing, and their applications in medicine and neuroscience. I hope to study for a Ph.D. in this field and make meaningful contributions both in theory and application. I choose this field because it is what I am good at and what I love. I learned to implement data structures and algorithms in C++ from the first grade of junior high school. I participated in the National Youth Computer Competition on behalf of our senior middle school and won the first prize of the National Olympiad in Informatics in Provinces(NOIP). I got a solid computing foundation through six years of training in programming and algorithm. However, I was not satisfied with the engineering-oriented training provided by the Department of Computer Science. I hoped that I could study algorithms and theories more deeply, for which I changed to the Department of Mathematics. Mathematics has indeed given me a lot of surprises. I am obsessed with the abstraction and simplicity of mathematical language.

As for why I am interested in medicine and neuroscience, my interest comes from pain. My family was hit hard by the diseases. I lost my father because he suffered from brainstem hemorrhage, and my mother-in-law was admitted to the hospital because of schizophrenia in 2020. I think that brain waves, CT, and MRI images, which are often referenced in medical research and diagnosis, are one-dimensional and two-dimensional signals. I strongly hope that I can contribute to solving the mechanism of mental illness using the tools from computational mathematics.

At present, I prefer to solve problems from the perspective of theoretical analysis and engage in cutting-edge work. So my first choice is to enter academia to discover and solve unknown problems. If I go to the industry, I hope to enter some laboratories or technical positions like algorithm engineers. Huawei's health monitoring group for Huawei Bands is an example. My research would be conducive to understanding how to make full use of medical data (physiological electrical signals, medical images), develop new medical devices, and contribute to human health. These cutting-edge technologies can also create potential economic benefits for enterprises.

Apart from academic research, I am curious about the world. I often travel alone or with close friends, and have been to Shanghai, Xiamen, Hong Kong, Changsha, and other places. I enjoy making plans by myself and exploring a city. To understand the real situation in Chinese developing areas, in the summer of 2019, I went to the Buyi ethnic minority settlement in Wangmo County, Guizhou Province on an expedition. We need to set up home visits, communicate and get information from needy parents and children there, for who we may find funders later on. The mountain road is very rugged, and the transportation is inconvenient. The children there are skilled in riding motorcycles to school, and they also drive us to their homes for home visits. Sometimes the schedule was tight, and we only ate two meals a day, while the children there are accustomed to having two meals a day. We stayed in their dormitory at night, and when we looked up, we fearfully saw moths covering the ceiling. We are sad to see that many children hesitate to continue studying because of family difficulties, while gratified to see that Chinese poverty alleviation policies indeed make a difference. Through seeing with my own eyes, I am closer to the reality of China.

Besides, I would like to communicate with foreign friends on the Internet. Cambly and Tandem are two Apps I like very much. On Cambly, I met foreign teachers from all over the world; on Tandem, I met foreign friends who were learning Chinese, and we exchanged languages.  For example, a little boy studying artificial intelligence at the University of Edinburgh always asks me questions in linear algebra. A Spanish brother studying for a Ph.D. in Physics in Texas, the United States told me that he learned Chinese because he found more and more Chinese materials in the physics literature and believed that Chinese was the future trend. Many points of view are very fresh to me, and I would like to travel the world and visit my friends in the future.

In daily life, I like drawing, reading, and all kinds of sports, like basketball, volleyball, and swimming. I believe that these hobbies are indispensable components of a well-rounded person, allowing me to love life even in the most difficult situations. During the COVID, I took classes online at home. My mother-in-law was admitted to the hospital due to schizophrenia and needs family members to take turns to accompany her. I recorded each prescription, observed her abnormal reactions after the medication, and communicated with the doctor. When a problem can not be solved, clinging to it makes people extreme and pessimistic. On the contrary, when I took up a paintbrush and painted cute little penguins,  drawing more than 100 pictures one by one, I could hide in my small world and forget the worries. The problem may remain, but the situation could stabilize, and I can get used to it and take responsibility steadily.

What is more, my hobbies brought me precious friendships. When I entered the university, our Geology Department had only freshmen on campus and no basketball team. Eager to play basketball, I co-founded a team with the Ocean Engineering Department and became the first captain. Contacting coaches, organizing training, registering for competitions, ...... Everything is a new experience. Composed of only freshmen, our team has completed many games, often with only 5 players and 1 substitute playing for four quarters under the scorching sun. We faced opponents composed of four grades, so every victory is hard-won and precious. Later, I handed over the captain to a brilliant forward in the team. This team continues to develop in Zhuhai, and I am proud of it. Team sports are intriguing because I like cooperation with teammates, the feeling of winning together, and the pure friendship among us. 

I am full of curiosity about the world, full of love for my field, and full of gratitude for life. Therefore, I will never stop exploring,  am not tired of research, and optimistically face the difficulties in life. I am proud of myself.




\end{document}