
\documentclass[UTF8]{beamer}
%\usepackage{subfigure}
\usepackage{url}   % 网页链接
\usepackage{subcaption}
%\usepackage[space,space,hyperref]{ctex}
%\usepackage{ctex} %可以改成中文版
\captionsetup{font={small}}
\usetheme{Warsaw}
\author{Tong Wu}
\title{Partially coherent ptychography}
%\institute{Sun yat-sen University, China}
\begin{document}
\frame{\titlepage}
\tableofcontents


\begin{frame}[c]\frametitle{Coherent model}


\begin{equation}
\label{basic}
f_{j}=\left|\mathcal{F}\left( \mathcal{S}_{j} u  \circ \omega \right)\right|^{2} 
\end{equation}

In a discrete setting, $u \in \mathbb{C}^{n}$ is a 2D image with $\sqrt{n} \times \sqrt{n}$ pixels, $\omega \in \mathbb{C}^{\bar{m}}$ is a localized 2D probe with $\sqrt{\bar{m}} \times \sqrt{\bar{m}}$ pixels.

$f_{j} \in \mathbb{R}_{+}^{\bar{m}}(\forall 0 \leq j \leq J-1)$ is a stack of phaseless measurements. Here $|\cdot|$ represents the element-wise absolute value of a vector, o denotes the elementwise multiplication, and $\mathcal{F}$ denotes the normalized 2 dimensional discrete Fourier transform. Each $\mathcal{S}_{j} \in \mathbb{R}^{\bar{m} \times n}$ is a binary matrix that crops a region $j$ of size $\bar{m}$ from the image $u$.


\end{frame}

\begin{frame}[c]\frametitle{Partially coherent model: a general one}
\framesubtitle{Thibault, Pierre, and Andreas Menzel. "Reconstructing state mixtures from diffraction measurements." Nature 494.7435 (2013): 68-71.}



\structure{Blind ptychography model + quantum state tomography\footnote{\url{https://homepage.univie.ac.at/reinhold.bertlmann/pdfs/T2_Skript_Ch_9corr.pdf} Many symbols in quantum mechanics are included here.}.}

Phobe $w$ is assumed to be in mixed state to represent partially coherent effect.

\begin{equation}
\label{sep} 
\begin{aligned}
&\mbox{Find } u, r \mbox{ othogonal $w_k$   }s.t. \\
&f_{p c, j}=\sum_{k=1}^r \left|\mathcal{F}\left( \mathcal{S}_{j} u \circ \left(\omega_k\right) \right)\right|^{2}  
\end{aligned}
\end{equation}


\end{frame}

\begin{frame}[c]\frametitle{Algorithm —— ePIE}



\noindent \textbf{Step1: Fourier magnitude projection}

We have used the shorthand notation $\tilde{\phi}=\mathcal{F}[\phi]$. Taking the form
$$
\begin{aligned}
\psi_{j \mathbf{x}}^{(k)} &=\bar{\Pi}_{F}\left(\phi_{j \mathbf{x}}^{(0)}, \phi_{j \mathbf{x}}^{(1)}, \ldots, \phi_{j \mathbf{x}}^{\left(r=M_{\mathrm{p}}\right)}\right) \\
&=\mathcal{F}^{-1}\left[\sqrt{I_{j \mathbf{q}}} \frac{\tilde{\phi}_{j \mathbf{q}}^{(k)}}{\sqrt{\sum_{k}\left|\tilde{\phi}_{j \mathbf{q}}^{(k)}\right|^{2}}}\right]
\end{aligned}
$$
again with
$$
\tilde{\phi}_{j \mathbf{q}}^{(k)}=\mathcal{F}\left[P_{\mathbf{x}}^{(k)} O_{\mathbf{x}-\mathbf{x}_{j}}\right]
$$


\end{frame}
\begin{frame}

\noindent \textbf{Step2: Overlap projection}

$$
\begin{aligned}
&O_{\mathbf{x}}=\frac{\sum_{k} \sum_{j} P_{\mathbf{x}+\mathbf{x}_{j}}^{(k) *} \psi_{j, \mathbf{x}+\mathbf{x}_{j}}^{(k)}}{\sum_{k} \sum_{j}\left|P_{\mathbf{x}+\mathbf{x}_{j}}^{(k)}\right|^{2}} \\
&P_{\mathbf{x}}^{(k)}=\frac{ \sum_{j} O_{\mathbf{x}-\mathbf{x}_{j}}^{*} \psi_{j \mathbf{x}}^{(k)}}{ \sum_{j}\left|O_{\mathbf{x}-\mathbf{x}_{j}}\right|^{2}}
\end{aligned}
$$
solved numerically by applying them sequentially for a few iterations.


\end{frame}

\begin{frame}{Outcome}

%\begin{figure}[H]   % 必须要有[h]否则插入的图片都在文字前面
%    \centering  % 图像居中
%    \includegraphics[width=6cm]{outcome/clipboard.png}%[]里可以指定影像大小
%    \caption{Probes $w_k$ and $u$}    % 图名
%    \label{1}  % 用于内部引用的图名
%\end{figure}
\end{frame}


\begin{frame}\frametitle{Another form}


Denote $O_j \in C^{\bar{m} \times \bar{m}}$ as a (diagonal) matrix to represent linear transform to $w$, s.t. $\mathcal{S}_{j} u \circ \omega = O_j w$. Denote $f_q^* \in C^{1 \times \bar{m}}$ as a row vector  constructed from Fourier transform $\mathcal{F}$, to represent projection on frepuency element. Construct measurement matrix $ \mathcal{I}_{j \mathbf{q}} = O_j^*f_qf_q^*O_j$ and density matrix $\rho$, we get another form(actually a natural one in quantum state tomography) of the model:



\begin{equation}
\label{lift}
\begin{aligned}
&\mbox{Find } O_j,\rho,s.t.\\
&f_{pc,j}(q) = Tr(\mathcal{I}_{j \mathbf{q}} \rho )\\
&\rho \mbox{ is positive semi-definite, with rank}\leq r 
\end{aligned}
\end{equation}

\end{frame}

\begin{frame}\frametitle{Derivation}

 Simple calculation process:
$$
f_{pc,j}(q) = |f_q^*O_j w|^2 = (f_q^*O_j w)^*(f_q^*O_j w) = w^*(O_j^*f_qf_q^*O_j)w
$$
$$
=Tr[w^*(O_j^*f_qf_q^*O_j)w]=Tr[(O_j^*f_qf_q^*O_j) (ww^*)]
$$
$$
=Tr(  \mathcal{I}_{j \mathbf{q}} \rho )
$$


When $w$ is in pure state(a vector in Hilbert space), $\rho=ww^*$ is a rank-one matirx. In partially coherent case, \textbf{we use mixed state to model $w$}.  Mixed state is represented by \textbf{generalizing the density matrix to one with higher rank}: 
$$
\rho = \sum_{k=1}^{r} w_k w_k^*
$$
$$
 f_{pc,j}(q) = \operatorname{Tr} \mathcal{I}_{j \mathbf{q}} \rho
 = \operatorname{Tr}[ \mathcal{I}_{j \mathbf{q}}  \sum_{k=1}^{r} w_k w_k^*]
=
\sum_{k=1}^r w_k^*\mathcal{I}_{j \mathbf{q}} w_k 
=
\sum_{k=1}^r |f_q^*O_j w_k|^2 
$$

And that is exactly \eqref{sep}$
 f_{pc,j}=\sum_{k=1}^r \left|\mathcal{F}\left( \mathcal{S}_{j} u \circ \left(\omega_k\right) \right)\right|^{2}  
$. 
\end{frame}
\begin{frame} \frametitle{Further explaination about mixed state}
 Fow example, with probability 0.5 in state $\psi_1$ and 0.5 in $\psi_2$ ($\psi_1$ and $\psi_2$ are not neccesarily orthogonal here). Now $w$ can no longer be represented by a vector(ps. $w \neq p_1\psi_1 + p_2 \psi_2$, the latter is still a determined pure state vector).
 
 $$
 \rho = \sum_k p_k \psi_k \psi_k^*
 $$
 
 
 
 Easy to find $\rho$ is a positive semi-definite matrix, we can decompose $\rho$ using spectral theorem, with $r$(rank of $\rho$) othogonal state $w_k$:
 
 $$
 \rho = \sum_{k=1}^{r} w_k w_k^*
 $$
 
\end{frame}
\begin{frame} \frametitle{Related work}
\framesubtitle{Fannjiang, Albert, and Thomas Strohmer. "The numerics of phase retrieval." Acta Numerica 29 (2020): 125-228. 6.3. Low-rank phase retrieval problems.}

The phase retrieval problem has a natural
generalization to recovering low-rank positive semidefinite matrices.
More precisely, we are concerned with the task of reconstructing a finite-dimensional quantum mechanical system which is fully characterized
by its density operator $\rho$ (an n x n positive semidefinite matrix with trace one.)
\framesubtitle{Fannjiang, Albert, and Thomas Strohmer. "The numerics of phase retrieval." Acta Numerica 29 (2020): 125-228. 6.3. Low-rank phase retrieval problems.}




\end{frame}

\end{document}
